\section{Fragen zu Vorlesung 1}
\subsection{Welche Instrumente hat die EU zur Reduktion der Handelshemmnisse im Handelsraum?}
\begin{outline}
  \1 Harmonisierungs- Standardisierungsdokumente
  \1 Normen z.B. EN Normen
  \1 \textbf{CE-Kennzeichnung}
\end{outline}

\subsection{Welche technischen Richtlinien kennen Sie im Bereich Elektrotechnik/Elektronik?}
\begin{outline}
  \1 LVD (Low Voltage Directive)
  \1 RED (Radio Equipment Directive)
  \1 EMC {Electromagnetic compatibility}
  \1 Medical devices
  \1 Maschinen-Richtlinie
\end{outline}

\subsection{Was sind horizontale und was sind vertikale Richtlinien? (Zeichnung, Zuordnung, Beschreibung)}
\textbf{Horizontale Richtlinien}\p
Querschnittsmaterie. Sie gelten für viele (alle) Produkte\p
%
\textbf{Vertikale Richtlinien}\p
Richtlinien für definierte spezielle Produkte (z.B. Medizin, KFZ).

\subsection{Unter welche Richtlinien müsste ein Produkt aus Elektromotor mit integriertem Netwechselrichter fallen?}
\begin{outline}
  \1 EMV
  \1 Maschinen-Richtlinie
  \1 LVD
\end{outline}

\subsection{Richtlinien}
\subsubsection{Was regelt die EMV Richtlinie?}
Regelt die Kompatibilität eines Geräts. D.h. Störfestigkeit und Störemissionen

\subsubsection{Was regelt die LVD Richtlinie?}
Regelt die Sicherheit, wie z.B. Isolation, Brand, mechanische Gefährdung, Strahlung

\subsubsection{Was regelt die RED Richtlinie?}
Regelt Sicherheit und Kompatibilität aller Produkte mit Funkschnittstellen (WLAN, BT, NF, ISM)

\subsection{Sind harmonisierte Normen verpflichtend anzuwenden?}
Nein, Harmonisierte Normen sind nicht zwingend vorgeschrieben

\subsection{Wer darf harmonisierte Normen erstellen?}
\begin{outline}
  \1 \textbf{Allgemein:} CEN
  \1 \textbf{Elektrotechnik:} CENELEC
  \1 \textbf{Telekommunikationssektor:} ETSI
\end{outline}

\subsection{Was bedeutet harmonisierte Norm?}
Harmonisierte Normen sind ein Mindeststandard und beschreiben somit die grundlegenden Anforderungen für die, von ihnen erfassten Produkte.\p
Harmonisierte Normen spiegeln den allgemein anerkannten Stand der Technik im Bezug auf die elektromagnetische Verträglichkeit in der EU wider.

\subsection{Welche Arten von Normen sind ihnen bekannt? Beschreiben Sie sie kurz.}
\begin{outline}
  \1 \textbf{Basic-Standard (Grundnorm)}
    \2 Beschreibt: Phänomen, Prüfgenerator, erforderliche Prüfaufbauten
    \2 Keine Angaben über Limits
  \1 \textbf{Generic-Standard (Fachgrundnorm)}
    \2 Kommen zur Anwendung, wenn keine Produktnorm zur Verfügung steht
  \1 \textbf{Product-Standard (Produktnormen)}
    \2 Anforderungen bestimmter Produkte hinsichtlich:
      \3 Betrieb
      \3 Messung
      \3 Bewertung der Funktionsstörungen
    \2 Vorrang vor Fachgrundnorm
    \2 Können besondere Grenzwerte oder veränderte Prüfungen beschreiben
\end{outline}

\pagebreak