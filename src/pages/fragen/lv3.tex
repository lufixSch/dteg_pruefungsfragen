\section{Fragen zu Vorlesung 3}

\subsection{Welches Frequenzspektrum hat ein sinusförmiges, welchess ein Pulsförmiges Signal?}
\begin{itemize}
  \item \textbf{Sinus:} Einzelner Puls bei der Frequenz des Signals
  \item \textbf{Puls:} Hoher Puls bei der Frequenz des Signals. Langsam absinkende Pulse bei Vielfachen der Grundfrequenz. (Höherer Puls bei ungeraden Vielfachen)
\end{itemize}

\subsection{Welche Signalform generiert die wenigsten Störemissionen?}
Ein Sinussignal, da es nur eine Frequenz besitzt.

\subsection{Wann ist grundsätzlich mit Störemissionen von elektronischen Geräten zu rechnen?}


\subsection{Erklären Sie das Quellen- Senkenmodell der EMV?}

\begin{itemize}
  \item \textbf{Störquelle:} Ursache von Störungen
  \item \textbf{Störsenke:} Wird beeinflusst durch Störungen
  \item \textbf{Koppelpfad:} Störende Verbindung zwischen Quelle und Senke
\end{itemize}

\subsection{Welche Arten von Koppelpfaden kennen Sie? Welche Maßnahmen können Sie zur Reduktion der Koppelpfade treffen?}
\begin{itemize}
  \item Galvanische Kopplung
  \item Nahfeldkopplung
  \item Einstrahlungsverkopplung
  \item kapazitive Kopplung
  \item induktive Kopplung
\end{itemize}

\subsection{Erklären Sie eine galvanische Kopplung im Detail. Welche Maßnahmen kennen Sie um eine galvanische Kopplung zu reduzieren}
Entsteht bei einem gemeinsamen Rückleiter (Gemeinsame Masseleitung). Dur Leitungswiderstand entsteht Störspannung, die in die Störsenke gelangt.\p
\textbf{Maßnahmen}\p
\begin{itemize}
  \item Nur gemeinsamen Massepunkt
\end{itemize}


\subsection{Was wird unter Nahfeldkopplung verstanden? Welche dieser Kopplungen kennen Sie?}


\subsection{Erklären Sie eine kapazitive Kopplung. Wie kommt diese zustanden? Mit welchen Maßnahmen kann Sie reduziert werden}
Tritt auf durch parasitäre Kapazitäten. Verkopplung einzelner Elemente durch parasitäre Kapazitäten (z.B. dadurch, dass Leitungen parallel verlaufen). Tritt nur dann auf, wenn sich das Potential der Elemente unterscheidet.\p
\textbf{Maßnahmen}
\begin{itemize}
  \item Räumliche Trennung
  \item Reduktion der Leitungslängen
  \item Einbringen eines Massestreifen zwischen den Leitungen
\end{itemize}

\subsection{Erklären Sie eine induktive Kopplung. Wie kommt diese zustande? Welche Maßnahmen kennen Sie um diese zu reduzieren?}
Tritt auf durch (parasitäre) Induktivitäten. Die Induktivitäten einer Stromschleife induziert Strom in eine andere Stromschleife.\p
Die Kopplung wird beschrieben über den \textbf{Koppelfaktor}. Dies ist ein Faktor welcher angibt, wie viel Prozent des Störenden Signals auf die Schaltung wirkt. Die Reduzierung des Koppelfaktors reduziert den Einfluss der induktiven Kopplung.\p
\textbf{Maßnahmen}
\begin{itemize}
  \item Räumliche Trennung
  \item Abstrahlende Fläche, aufnehmende Fläche reduzieren (Leitungen kurz halten, Hin- und Rückleiter parallel zueinander verlegen)
\end{itemize}

\subsection{Was ist eine Strahlungskopplung? Wie kommt diese zustande? Welche Maßnahmen kennen Sie um diese zu reduzieren?}
Störung durch einstrahlende Felder (z.B. Funk).

\subsection{Wie können Störquellen grundsätzlich kategorisiert werden?}
\textbf{Vorkommen}
\begin{itemize}
  \item Dauernde Einwirkung
  \subitem Mobilfunk
  \subitem Energieversorgung (Stromrichter, Schaltnetzteile) benachbarter Geräte
  \item Einmalige Einwirkung
  \subitem Blitze (Surge)
  \subitem Elektrostatische Entladung (ESD)
  \subitem Schaltvorgänge (EFT-Burst)
  \subitem Spannungseinbrüche
  \subitem Radar
\end{itemize}
%
\textbf{Auftreten}
\begin{itemize}
  \item Leitungsgeführt
  \subitem ESD
  \subitem Burst
  \subitem Surge

  \item Eingestrahlt
  \subitem Mobilfunk
\end{itemize}

\subsection{Wie entstehen Schmalbandstörer?}

\subsection{Wie entstehen Breitbandstörer?}

\subsection{Welche schmalbandigen Störungen kennen Sie?}

\subsection{Welche breitbandigen Störungen kennen Sie?}

\subsection{Was versteht man unter den EMV Prüfungen Surge, Burst, ESD und HF gestrahlt? Welche Störungen werden damit simuliert?}
\begin{itemize}
  \item \textbf{Surge:} Spannungs Impulse (50MHz)
  \item \textbf{Burst:} Schaltvorgänge (100MHz)
  \item \textbf{ESD:} Elektrostatische Entladung (100MHz)
  \item \textbf{HF:}
\end{itemize}