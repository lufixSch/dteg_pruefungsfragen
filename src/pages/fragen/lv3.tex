\section{Fragen zu Vorlesung 3}

\subsection{Welches Frequenzspektrum hat ein sinusförmiges, welchess ein Pulsförmiges Signal?}\label{sec:lv3:freq_spekt}
\begin{outline}
  \1 \textbf{Sinus:} Einzelner Puls bei der Frequenz des Signals
  \1 \textbf{Puls:} Hoher Puls bei der Frequenz des Signals. Langsam absinkende Pulse bei Vielfachen der Grundfrequenz. (Höherer Puls bei ungeraden Vielfachen)
\end{outline}

\subsection{Welche Signalform generiert die wenigsten Störemissionen?}
Ein Sinussignal, da es nur eine Frequenz besitzt.

\subsection{Wann ist grundsätzlich mit Störemissionen von elektronischen Geräten zu rechnen?}
Störemissionen treten auf wenn sich auf der Platine elektrische- / magnetische (Wechsel-)Felder bilden.

\begin{outline}
  \1 Jedes Gerät \(\rightarrow\) erzeugt elektromagnetische Störung (Elektronenbewegung, Potentialunterschiede)
  \1 Störungsaussendung = leitungsgeführt oder gestrahlt
\end{outline}

\subsection{Welche Maßnahmen können gesetzt werden um breitbandige Störemissionen von Signalen zu reduzieren?}
Glätten der Impulse durch beispielsweise Kapazitäten.

\subsection{Warum entstehen Oberwellenströme bei einem Betrieb von Schaltnetzteilen?}
Durch das Schalten des Transistors. Dadurch entstehen große Spannungssprünge am Knoten nach dem Transistor welche die parasitären Kapazitäten laden bzw. entladen.

\subsection{Erklären Sie das Quellen- Senkenmodell der EMV?}
\begin{outline}
  \1 \textbf{Störquelle:} Ursache von Störungen
  \1 \textbf{Störsenke:} Wird beeinflusst durch Störungen
  \1 \textbf{Koppelpfad:} Störende Verbindung zwischen Quelle und Senke
\end{outline}

\subsection{Welche Arten von Koppelpfaden kennen Sie? Welche Maßnahmen können Sie zur Reduktion der Koppelpfade treffen?}
\begin{outline}
  \1 Galvanische Kopplung
  \1 Nahfeldkopplung
  \1 Einstrahlungsverkopplung
  \1 kapazitive Kopplung
  \1 induktive Kopplung
\end{outline}

\subsection{Erklären Sie eine galvanische Kopplung im Detail. Welche Maßnahmen kennen Sie um eine galvanische Kopplung zu reduzieren}
Entsteht bei einem gemeinsamen Rückleiter (Gemeinsame Masseleitung). Durch Leitungswiderstand entsteht Störspannung, die in die Störsenke gelangt.\p
\textbf{Maßnahmen}
\begin{outline}
  \1 Nur gemeinsamen Massepunkt
\end{outline}


\subsection{Was wird unter Nahfeldkopplung verstanden? Welche dieser Kopplungen kennen Sie?}
Störungen, welche durch niederfrequente elektrische oder magnetische Felder entstehen (bis maximal \(100MHz\)). Oft entstehen diese auf der selben Platine/im selben Gehäuse, wie beispielsweise bei \textbf{kapazitiver} und \textbf{induktiver} Kopplung.


\subsection{Erklären Sie eine kapazitive Kopplung. Wie kommt diese zustanden? Mit welchen Maßnahmen kann Sie reduziert werden}
Verkopplung einzelner Elemente durch parasitäre Kapazitäten (z.B. dadurch, dass Leitungen parallel verlaufen). Tritt nur dann auf, wenn sich das Potential der Elemente unterscheidet.\p
\textbf{Maßnahmen}
\begin{outline}
  \1 Räumliche Trennung
  \1 Reduktion der Leitungslängen
  \1 Einbringen eines Massestreifen zwischen den Leitungen
\end{outline}

\subsection{Erklären Sie eine induktive Kopplung. Wie kommt diese zustande? Welche Maßnahmen kennen Sie um diese zu reduzieren?}
Tritt auf durch (parasitäre) Induktivitäten. Die Induktivitäten einer Stromschleife induziert Strom in eine andere Stromschleife.\p
Die Kopplung wird beschrieben über den \textbf{Koppelfaktor}. Dies ist ein Faktor welcher angibt, wie viel Prozent des Störenden Signals auf die Schaltung wirkt. Die Reduktion des Koppelfaktors reduziert den Einfluss der induktiven Kopplung.\p
\textbf{Maßnahmen}
\begin{outline}
  \1 Räumliche Trennung
  \1 Abstrahlende Fläche, aufnehmende Fläche reduzieren (Leitungen kurz halten, Hin- und Rückleiter parallel zueinander verlegen)
\end{outline}

\subsection{Was ist eine Strahlungskopplung? Wie kommt diese zustande? Welche Maßnahmen kennen Sie um diese zu reduzieren?}
Störung durch einstrahlende höherfrequente Fernfelder (z.B. Elektromagnetische Feldkopplung \textit{LTE, 5G, Radiowellen}).\p
%
\textbf{Maßnahmen}
\begin{outline}
  \1 Verwendung eines geschirmten Gehäuses.
\end{outline}

\subsection{Wie können Störquellen grundsätzlich kategorisiert werden?}
\textbf{Vorkommen}
\begin{outline}
  \1 Dauernde Einwirkung
  \2 Mobilfunk
  \2 Energieversorgung (Stromrichter, Schaltnetzteile) benachbarter Geräte
  \1 Einmalige Einwirkung
  \2 Blitze (Surge)
  \2 Elektrostatische Entladung (ESD)
  \2 Schaltvorgänge (EFT-Burst)
  \2 Spannungseinbrüche
  \2 Radar
\end{outline}
%
\textbf{Auftreten}
\begin{outline}
  \1 Leitungsgeführt
  \2 ESD
  \2 Burst
  \2 Surge

  \1 Eingestrahlt
  \2 Mobilfunk
\end{outline}

\subsection{Wie entstehen Schmalbandstörer?}
Werden von \textbf{periodischen} bzw. getakteten Quellen erzeugt. Schmaler Frequenzbereich
\textbf{Beispiele}
\begin{outline}
  \1 Rundfunkt
  \1 Schaltnetzteile
  \1 Prozessoren
\end{outline}

\subsection{Wie entstehen Breitbandstörer?}
Werden von nicht periodischen bzw. einmaligen \textbf{Störimpulsen} erzeugt. Großer Frequenzbereich bis in \m{GHz} Bereich
\textbf{Beispiele}
\begin{outline}
  \1 ESD
  \1 Blitzentladungen
  \1 Schaltvorgänge
\end{outline}

\subsection{Welche schmalbandigen Störungen kennen Sie?}
\begin{outline}
  \1 Mobilfunk
  \1 Energieversorgung (Stromrichter, Schaltnetzteile) benachbarter Geräte
\end{outline}

\subsection{Welche breitbandigen Störungen kennen Sie?}
\begin{outline}
  \1 Blitze (Surge)
  \1 Elektrostatische Entladung (ESD)
  \1 Schaltvorgänge (EFT-Burst)
  \1 Spannungseinbrüche
  \1 Radar
\end{outline}

\subsection{Was versteht man unter den EMV Prüfungen Surge, Burst, ESD und HF gestrahlt? Welche Störungen werden damit simuliert?}
\begin{outline}
  \1 \textbf{Surge:} Spannungs Impulse (50MHz)
  \1 \textbf{Burst:} Schaltvorgänge (100MHz)
  \1 \textbf{ESD:} Elektrostatische Entladung (100MHz)
  \1 \textbf{HF:} High Frequency
\end{outline}

\subsection{Wie sieht eine beispielhafte Einteilung von Störsenken anhand der Art der Ausfallerscheinung aus? (FSPC = Functional Status Performance Classes)}
\begin{outline}
  \1 Klasse A
    \2 Keine Reaktion auf Störgrößen
    \2 Alle relevanten Parameter befinden sich immer in ihrem übeblichen Toleranzbereich
    \2 Seltene singuläre Ereignisse \m{\rightarrow} z.B. Blitzeinschlag \m{\rightarrow} abrutschen in Klasse B
  \1 Klasse B
    \2 Reaktion des Geräts fällt außerhalb des üblichen Toleranzbereichs
    \2 Nach Abklingen der Störgröße \m{\rightarrow} Rücksprung in Normalzustand
    \2 Nutzen: nicht sicherheitsrelevante Anwendungen \m{\rightarrow} wenn Störgröße dauerhaft vorherrschend \m{\rightarrow} Klasse B nicht zulässig (Funktion dann dauerhaft nicht möglich)
  \1 Klasse C
    \2  Reaktion des Geräts fällt außerhalb des üblichen Toleranzbereichs
    \2 Nach Abklingen der Störgröße \m{\rightarrow} kein Rücksprung in Normalzustand \m{\rightarrow}Benutzereingriff erforderlich
    \2 Gedacht für nicht-sicherheitsrelevante Funktionen \m{\rightarrow} und Phänomene bei welchen ein Hardwarereset selten auftreten
  \1 Klasse D
    \2 System wird nach Störeinwirkung dauerhaft beschädigt \m{\rightarrow} muss von qualifizierter Person repariert werden
    \2 In \textbf{keinem} System zulässig es sei denn \m{\rightarrow} Komfortfunktion (Leselicht im Auto) oder selten auftretende Phänomene
  \1 Klasse FS
    \2 ,,Fail Safe''
    \2 System fällt aufgrund anliegender Störgröße in sicheren Zustand zurück (abschalten, neustarten)
    \2 vergleichbar mit C, jedoch muss Zustand stärker definiert sein \m{\rightarrow} Anforderung höher
    \2 große Maschinen, nur von qualiziertem Personal bedient, darf auch keinen Fall unmotivierte Bewegungen aufgrund von Gewicht oder Motorkraft ausführen
\end{outline}

\pagebreak